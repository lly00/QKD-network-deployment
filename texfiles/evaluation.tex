$x(t)$ represents the qubit rate ratio of the quantum channel on each path, and $\lambda(f)$ represents the qubit rate that the quantum key network provide for S-D pair $f$.  $w(f)$ represents the minimum qubit rate required by $f$. Our goal is to maximum the total throughput of the quantum key network.
\begin{equation*}
    \begin{aligned}
    \max \ \ \sum_{f \in F}\sum_{t \in T_f}{z_f^t \cdot x(t) \cdot \lambda(f)} \\
    \end{aligned}
	\end{equation*}
	\begin{equation}\label{eq:profit}
	S.t.\begin{cases}
    \sum_{t \in T_f}z_f^t \le h, & \forall f \in F \\
     \sum_{t \in T_f} x(t) = 1, & \forall f \in F \\
     \lambda(f) \ge w(f), & \forall f \in F \\
     \sum_{f \in F}\sum_{q \in t:t \in T_f}{z_f^t \cdot \lambda(f) \cdot x(t)} \le y(q), & \forall q \in Q \\
     \sum_{f \in F}\sum_{e \in t:t \in T_f}{z_f^t \cdot v(f) \cdot x(t)} \le c(e), & \forall e \in E \\
     z^f_t \in \{0,1\} \\
     x(t) \in [0,1]

	\end{cases}
\end{equation}
The first set of inequalities denotes that we can choose at most $h$ paths for $f$. The second set of inequalities denotes that all selected paths can completely transmit qubits. The third set of inequalities denotes that the quantum key network can satisfy the minimum requirement of $f$. The fourth and fifth set of inequalities denotes the quantum channel load and data link load must not exceed their capacity.

\begin{equation*}
    \begin{aligned}
    \max \ \ \sum_{f \in F}\sum_{t \in T_f}{x(t) \cdot \lambda(f)} \\
    \end{aligned}
	\end{equation*}
	\begin{equation}\label{eq:profit}
	S.t.\begin{cases}
     \sum_{t \in T_f} x(t) = 1, & \forall f \in F \\
     \lambda(f) \ge w(f), & \forall f \in F \\
     \sum_{f \in F}\sum_{t:q \in t}{\lambda(f) \cdot x(t)} \le y(q), & \forall q \in Q \\
     \sum_{f \in F}\sum_{t:e \in t}{\cdot v(f) \cdot x(t)} \le c(e), & \forall e \in E \\
     x(t) \ge 0

	\end{cases}
\end{equation}


There are chances that 