
\subsection{QKD Network Model}
Consider three sets in a quantum key distribution(QKD) network: a QKD endpoint set, $N=\left\{n_1,n_2,...,n_m\right\}$, with $m=\left|N\right|$; a QKD trusted relay set $Q$ and a set of links between endpoints and relays, such as optical fibers, denoted as $E$. We perform QKD between two nodes(endpoints or relays) through a quantum channel built on a physical link of $E$.


\subsection{QKD Network Topology Control Model}
Under the metropolitan area network, some organizations will deploy some quantum key distribution endpoints(QKD Endpoint) for secure communications. Direct communication may be not supported because QKD endpoints can be far apart, so we need to deploy some trusted QKD relays to implement quantum key distribution. To achieve the security requirements of trusted relays, we can not choose random places to deploy relays. Based on the existing traditional network topology, the locations of trusted relays can coincide with some key locations(such as core switches) in a candidate location set, which is denoted as $P=\left\{p_1,p_2,...,p_k\right\}$, with $k=\left|P\right|$. QKD network topology control(QNTC) relies on the deployment of trusted relays and optical fibers to connect these QKD endpoints, and our goal is to reduce the network deployment cost(the number of deployed relays)

In this paper, we propose the QNTC problem. We assume that each endpoint pair needs at least one quantum channel, so there must be at least one available path for each endpoint pair that satisfies quantum constraints. There are mainly two constraints. One is the quantum bit(qubit) capacity of a relay, which leads to the limited number of quantum channels passing through this relay. The other one is the link length constraint, that is, the length of any link must not exceed a given threshold, otherwise the quantum signals will not be measured correctly.

Let $\Upsilon$ denote quantum capacity of one relay $q \in Q$ and $F$ be the S-D pair set. We use $x_i$ to denote that whether to deploy relay at position $p_i \in P$ ($x_i = 1 $) or not ($x_i = 0$). Since there should be at least one quantum channel between each S-D pair, there must be at least one available path(includes relays and endpoints) where the physical length of each link is less than $L_{max}$. For any S-D pair $f$, we can find all the candidate paths that satisfy the link length constraint, denoted as $T_f$. $z_j^f$ denotes that if we use the candidate path $t_j^f$ chosen from $T_f$ or not. Each path through $p_i$ will occupy two qubits of $p_i$ if the path is chosen for S-D pair $f$, so the total number of occupied qubits of relay at $p_i$ is denoted as $\sum_{t^f_j: p_i \in t^f_j}{2*z^f_j}$.

To find candidate path set $T_f$ for each $f$, firstly we construct a graph with given node set consists of endpoint set $N$ and candidate location set $P$. Then we add a link between any two nodes if the physical length of link between them is less than $L_{max}$. After construction, we can run an algorithm(such as Yen's\cite{Multipaths}) on this graph to get multiple paths $T_f$ for each S-D pair $f$.

The QNTC problem is denoted as below. Our goal is to minimize the number of deployed relays.

%{\small
%	\begin{equation*}
%	\min \ \  \sum_{p_i \in P}{x_i}
%	\end{equation*}
%	\begin{equation}\label{eq:profairness:int_formulation}
%	{S.t.}\begin{cases}
%	 \sum_{f \in F}{2*y^f_i} \le \Upsilon , & \forall p_i \in P \\
%     y^f_i \le x_i, & \forall f \in F, \forall p_i \in P \\
%	 z_j^f \le y^f_i, & \forall p_i \in t_j^f, \forall t_j^f \in T_f, \forall f \in F \\
%     \sum_{t_j^f \in T_f}{z_j^f} = 1, & \forall f \in F \\
%
%	 x_i \in \{0, 1\}\\
%     y^f_i \in \{0, 1\}\\
%     z^f_j \in \{0,1\}
%
%	\end{cases}
%	\end{equation}

{\small
	\begin{equation*}
	\min \ \  \sum_{p_i \in P}{x_i}
	\end{equation*}
	\begin{equation}\label{eq:profairness:int_formulation}
	{S.t.}\begin{cases}
	 \sum_{t^f_j: p_i \in t^f_j}{2*z^f_j} \le \Upsilon , & \forall p_i \in P \\
     \sum_{t_j^f \in T_f}{z_j^f} = 1, & \forall f \in F \\
     z_j^f \le x_i, & \forall p_i \in t_j^f, \forall t_j^f \in T_f, \forall f \in F \\
	 x_i \in \{0, 1\}\\
     z^f_j \in \{0,1\}

	\end{cases}
	\end{equation}
The first set of inequalities indicates the occupied qubits should not exceed the quantum capacity for each $p_i \in P$. The second set of inequalities denotes that each S-D pair $f$ must have one available path. The third set of inequalities means that we should deploy relays in all the positions that path $t_j^f$ passes if $z^f_j = 1$.
%If we denote $Q$ as the available location set for relay deployment, then we have
%{\small
%    \begin{equation*}
%	\min \ \  \alpha \sum_{q \in Q}{y^q} + \beta \sum_{e \in E}{l(e)}
%	\end{equation*}
%	\begin{equation}\label{eq:profairness:int_formulation}
%	{S.t.}\begin{cases}
%	 \sum_{n_i \in N}\sum_{n_j \in N,j \neq i}{ x^q_{i,j}} \le \Upsilon, & \forall q \in Q \\
%	 \sum_{p \in \phi (q)}{ x^p_{i,j}} = 2, & \forall q \in Q,\forall n_i,n_j \in N \\
%     \sum_{p \in \phi (n_i)}{ x^p_{i,j}} = 1, & \forall n_i,n_j \in N \\
%	 l(e) \le L_{max}, & \forall e \in E\\
%
%	x^q_{i,j} \le y^q. & \forall q \in Q\\
%	x^q_{i,j} \in \{0, 1\}\\
%    y^q \in \{0, 1\}
%	\end{cases}
%	\end{equation}}
%where $y^q$ denotes that if we choose $q$ as a location for a quantum relay deployment or not.

{\small
	\begin{equation*}
	\min \ \  \sum_{j=1}^{n}{c_{j}x_{j}}
	\end{equation*}
	\begin{equation}\label{eq:profairness:01}
	{S.t.}\begin{cases}
	 \sum_{j=1}^{n}{a_{ij}x_j} \ge b_i\\
	 x_j \in \{0, 1\}

	\end{cases}
	\end{equation}

\subsection{Problems}\label{sec:Design}
Problem 1: Would a endpoint perform as a relay? If so, the quantum capacity of endpoints should be considered.

Problem 2: Is there a constraint of number of channels between each pair of endpoints.





