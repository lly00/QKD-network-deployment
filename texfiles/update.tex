In this section, We first discuss scenarios where the quantum key rate requirements between endpoints change dynamically. Then we give solutions in these cases and propose an approximation routing algorithms.
\subsection{Dynamic Scenarios in QKD Network}
In practice, the S-D pair set $F$ may changes and the path width requirement $w_f$ for each S-D pair $f$ may changes too. When meeting these dynamic scenarios in QKD network, the number of quantum channels may exceed the capacity of a relay. At this time, we should re-select the path for every S-D pair, try to satisfy the quantum device constraint of each relay. However, considering devices are portable, when the quantum device constraints can not be satisfied anymore, we should update the quantum devices for each relay.


There are mainly three dynamic scenarios that will break constraints in QKD network. The first one is the change of S-D pair $F$. Considering that there are two quantum endpoints which were not communicating before want to perform quantum key distribution, we need to specify a path for them. The second is that the bandwidth required of one S-D pair $f$ changes. Under such circumstances, the default path for $f$ may need to be changed cause the quantum capacity of used relays may be not enough. The third dynamic scenario is the failure of quantum devices. Once the device used fails, $f$ may needs to select another path if the quantum bandwidth is not satisfied anymore. In more serious cases, a certain relay may fail completely or become untrustworthy due to some force majeure. At this time, we need to update the route as soon as possible, otherwise it will make QKD invalid for a long time, because the time to restore the relay can not be established.


\subsection{k-Splittalbe Routing in a QKD network}
After deploying a QKD network, an efficient routing mechanism will help to improve the network performance, such as high throughput of secret bits. We studies the throughput maximization problem of QKD by k-splittable routing(k-TMQN). Given an integer k $\ge$ 1, we assume that each S-D pair transmit secret bits through at most k different paths. The splittable routing scheme helps to improve the QKD network throughput by increasing the utilization of the quantum devices. However, the arbitrarily splittable scheme will lower the security level and increase the management difficulty of the QKD network. 

Efficient route selection depends on the current QKD network workload. We should dynamiclly update the routing schema so as to adapt the dynamic scenarios as stated before. Thus, we care for the current S-D pair set $F'$ instead of the long-term S-D pair set $F$ in section 3. The set $F'$ changes accordingly with the QKD network system running. Each S-D pair $f \in F$ will transmit secret bits through at most k permissible paths from source to destination. Suppose that we can transmit secret traffic load of $\lambda \cdot w(f)$ for each S-D pair $f$, where $\lambda$ is the throughput factor and $w(f)$ is the secret traffic demand of $f$. The k-TMQN problem will select at most k permissible paths for each S-D pair $f$, and allocate a feasible secret bit bandwidth on each path. Let $c(q)$ denote the capacity of relay $q$. To avoid congestion, the load on each relay should not exceed its capacity. The object is to maximize the network throughput factor $\lambda$.

We give the formulation of the k-TMQN problem as belows.
\begin{equation*}
\begin{aligned}
\max \ \ \lambda \\
\end{aligned}
\end{equation*}
\begin{equation}\label{eq:throughputmax}
S.t.\begin{cases}
\sum_{t \in T_f}z_f^t \le k, & \forall f \in F \\
\sum_{t \in T_f} x(t) \ge \lambda \cdot w(f), & \forall f \in F \\
\sum_{f \in F}\sum_{q \in t:t \in T_f}{z_f^t \cdot x(t) \cdot R(t,q)} \le c(q), & \forall q \in Q \\
z^f_t \in \{0,1\} & \forall f,t\\


\end{cases}
\end{equation}

Note that x(t) denotes the allocated bandwidth through each path $t$. The first set of inequalities ensures that each S-D pair at most select k paths. The second set of inequalities means that the achievable throughput of each S-D pair $f$ should not be less than $\lambda \cdot w(f)$, where $\lambda$ is throughput factor. The third set of inequalities represents that the total traffic load on relay $q$ should not exceed its key generation capacity.
\subsection{Algorithm for the k-TMQN problem}
In this section, we design an algorithm for QKD k-splittable routing with quantum device capacity constraints after the QKD network topology is decided in section 3. To deal with routing challenges in a QKD, the algorithm explores a permissible path set for each flow firstly, and selects at most $k$ permissible paths for routing between each pair of source and destination using randomized rounding mechanism in the second step. We proposed the approximate algorithm, called RSRQ(k) in the following part. 

In the first step, we use a recursive method to search a permissible path set $T_f$ for each S-D pair $f$.

In the second step, we constructs a linear program as a relaxation for the k-TMQN problem. As k-TMQN assumes that the traffic of each S-D pair will only go through at most k permissible paths, we formulate the following linear program $LP_2$ by relaxing this assumption. That is , the traffic of each $f$ can be forwarded through a set of permissible paths arbitrarily. The equation (\ref{eq:relaxed_tm}) formulates this linear program.

 \begin{equation*}
 \begin{aligned}
 \max \ \ \lambda \\
 \end{aligned}
 \end{equation*}
 \begin{equation}\label{eq:relaxed_tm}
 S.t.\begin{cases}
 \sum_{t \in T_f} x(t) \ge \lambda \cdot w(f), & \forall f \in F \\
 \sum_{f \in F}\sum_{q \in t:t \in T_f}{x(t) \cdot R(t,q)} \le c(q), & \forall q \in Q \\
	x(t) \ge 0, \forall t
 
 
 \end{cases}
 \end{equation}
 
The first set of inequalities denotes that the throughput of each S-D pair $f$ should not be less than $\lambda \cdot w(f)$, where $\lambda$ is the throughput factor and $w(f)$ is the secret bit rate demand between each S-D pair. The second set of inequalities ensures that the total traffic load on each relay $q$ should not exceed its key generation capacity $c(q)$. The third set of inequalities means that the traffic on any path is non-negative.


Since $LP_2$ is a linear program, we can solve it in polynomial time with a linear program solver. Assume that the optimal solution of $LP_2$ is $\widetilde{x}$ and the result of throughput factor is denoted as $\widetilde{\lambda}$. Since $LP_2$ is a relaxtion of the k-TMQN problem, $\widetilde{\lambda}$ is an upper-bound of throughput factor for k-TMQN. 

After solving the linear problem, we choose at most $k$ permissible paths from $T^{'}_{f}$ for each S-D pair $f$ with $x(t) > 0$ and allocate a feasible bandwidth $\hat{x}(t)$ for each selected path $t$. The detailed steps are described in Alg.({\ref{alg:RSRQ})

\begin{algorithm}[h]\label{RSRQ}
	\caption{RSRQ(k):Rounding-based k-Splitttable Routing in a QKD Network}
	\begin{algorithmic}[1]
		\STATE  {\bfseries Step 1: Initialization}
		\FOR{each S-D pair $f \in F$}
		\STATE Calculate the permissible path set $T_f$
		\ENDFOR
		\STATE  {\bfseries Step 2: Solving the Relaxed Problem}
		\STATE  Construct a linear optimization program $LP_2$ in Eq.\ref{eq:relaxed_tm}
		\STATE Obtain the optimal solutions $\widetilde{x}(t)$
		\STATE  {\bfseries Step 3: Selecting Paths for S-D pairs}
		\FOR{each S-D pair $f \in F$}
		\STATE $k' = k$ and $w'(f) = \widetilde{\lambda}w(f)$
		\WHILE{$max\{x(t)\} \ge \frac{w'(f)}{k'}$}
		\STATE Put path $t$ to set $T^{'}_f$ with $x(t) \ge \frac{w'(f)}{k'}$
		\STATE Set $\hat{x}(t) = \widetilde{x}(t), w'(f) = \widetilde{\lambda}w(f) - \sum_{t \in T^{'}_f} \widetilde{x}(t), k' = k - |T^{'}_f|$
		\ENDWHILE
		\STATE $T^{''}_{f} = T^k_f - T^{'}_f$
		\FOR {each $t \in T^{''}_{f}$}
		\STATE $y(t) = \frac{k' \cdot \widetilde{x}(t)}{w'(f)} = \frac{k' \cdot \widetilde{x}(t)}{\sum_{t \in T^{''}_f} \widetilde{x}(t)}$
		\ENDFOR
		\STATE Put all paths into k' kanpsacks with min-max weight. The ith path set in knapsack i is denoted as $T^i_f$;
		\FOR{each knapsack i}
		\STATE $z_i = \sum_{t \in T^i_f}y(t)$
		\STATE Randomly choose a path $t \in T^i_f$ for S-D pair $f$ with probability $\frac{y(t)}{z_i}$
		\STATE $\hat{x}(t) = \frac{w'(f)z_i}{k'}$
		\ENDFOR
		\ENDFOR
	\end{algorithmic}
\end{algorithm}


The algorithm selects at most k permissible paths  for each S-D pair and allocate feasible secret bandwidth of each path. For each S-D pair $f$, if $|T^k_f| \le k$, we just choose these paths for $f$, and set $\hat{x}(t) = \widetilde{x}(t)$ accordingly. If not, we divide $|T^k_f|$ into two sets, $T^{'}_{f}$ and$ T^{''}_{f}$.To construct the set  $T^{'}_{f}$, we set $k' = k$ and $w'(f) = \widetilde{\lambda}w(f)$ initially and then select a path with $x(t) \ge \frac{w'(f)}{k'}$ and update $\hat{x}(t) = \widetilde{x}(t), w'(f) = \widetilde{\lambda}w(f) - \sum_{t \in T^{'}_f} \widetilde{x}(t), k' = k - |T^{'}_f|$ in each iteration unless we can not find any path which meets the condition. After that, the second path set $T^{''}_{f}$ is determined to be $T^k_f - T^{'}_f$. For each path $t \in T^{''}_{f}$, we use another variable $y(t) = \frac{k' \cdot \widetilde{x}(t)}{w'(f)} = \frac{k' \cdot \widetilde{x}(t)}{\sum_{t \in T^{''}_f} \widetilde{x}(t)}$ ti to denote the weight of each path. In this way, the weight of each path  follows $0 < y(t) < 1$ and $\sum_{t \in T^{''}_f}y(t) = k'$. We solve the min-max knapsack problem by placing each path into k' knapsacks so as to minimize the total weight of all paths in each knapsacks. In the $i_{th}$ knapsack, the set of paths is denoted as $T^i_f$, and the total weight of the $i_{th}$ knapsacks is denoted as $z_i = \sum_{t \in T^i_f}y(t)$. At last, we choose one path from each path set $T^i_f$ and allocate a feasible bandwidth $\hat{x}(t) = \frac{w'(f)z_i}{k'}$. Path $t$ with has weight $y(t)$ will be chosen with probability $\frac{y(t)}{z_i}$. For the other paths in the set, we set $\hat{x}(t) = 0$.

We can give the following lemma.
\begin{lemma}
	After random rounding selection, each S-D pair $f$ has at most $k$ permissible paths with throughput $\widetilde{\lambda} \cdot w(f)$.
	
	\begin{proof}
		If $|T^k_f| \le k$, the algorithm simply choose those paths for $f$. Otherwise, we divide the set into two sets. Firstly we decide set $T^{'}_f$ in a loop operation. After that, the remaining bandwidth that needs to be provided and the number of paths that can be forwarded is decided as $w'(f) = \widetilde{\lambda}w(f) - \sum_{t \in T^{'}_f} \widetilde{x}(t)$ and $k' = k - |T^{'}_f|$, respectively. Then we choose $k'$ paths from $T^{''}_f$, with total throughput:
		\begin{equation}
		\sum_{t \in T^{''}_f} \hat{x}(t) = \sum_{T^i_f}\frac{w'(f)z_i}{k'} = \frac{w'(f)}{k'} \sum_{t \in T^{''}_f} y(t) = w'(f)
		\end{equation}
		So the total throughput of paths in both $ T^{'}_f$ and $ T^{''}_f$ is:
		\begin{equation}
		\sum_{t \in T^{''}_f} \hat{x}(t) + \sum_{t \in T^{'}_f} \hat{x}(t) = w'(f) + \sum_{t \in T^{'}_f} \widetilde{x}(t) = \widetilde{\lambda} \cdot w(f).
		\end{equation}
	\end{proof}
\end{lemma}


	
\subsection{Algorithm Performance Analysis}
We analyze the approximate performance of the proposed RSRQ(k) algorithm in this section. Like the previous proof, we calculate the expected key generation rate on each quantum relay and bound the probability that the key generation capacity will be violated. After solving the linear program $LP_2$, we get a fractional solution of the relaxed k-TMQN problem. By randomized rounding method, for each path $t \in T_f$, its required key generation rate on relay $q$ is denoted as a random variable $\theta_{f,t,q}$. Thus, the required key generation rate on relay $q$ for S-D pair $f$ is defined as a random varialbe $\theta_{f,q} = \sum_{t \in T_f}\theta_{f,t,q}$.
For each S-D pair $f$, if $t \in T^{'}_f$, then $\theta_{f,t,q} = \widetilde{x}(t)R(t,q)$. If $t \in T^{''}_f$, we have expected value of $\theta_{f,t,q}$, that is, $\mathbb{E}[\theta_{f,t,q}]= \frac{y(t)}{z_i} \cdot \frac{w'(f)z_i}{k'} \cdot R(t,q) = \frac{w'(f)y(t)}{k'} \cdot R(t,q) = \widetilde{x}(t) \cdot R(t,q)$. We also define $\theta^{''}_{f,q} = \sum_{t \in T^{''}_f}\theta_{f,t,q}$. Obviously, $\mathbb{E}[\theta^{''}_{f,q}]= \sum_{q \in t:t \in T^{''}_f}\widetilde{x}(t) \cdot R(t,q)$. $\delta$ is denoted as $max\{\sum_{f \in F}[\sum_{q \in t:t \in T^{''}_f}\widetilde{x}(t)\cdot R(t,q)]/c(q), \forall q \in Q\}$. Obviously $0 \le \delta \le 1$. The expected secret key generation rate on each relay is:
\begin{equation}
\begin{aligned}
	\mathbb{E}[\sum_{f \in F}\theta^{''}_{f,q}] &=  \sum_{f \in F} \mathbb{E}[\theta^{''}_{f,q}] \\
	&= \sum_{f \in F}\sum_{q \in t:t \in T^{''}_f}\widetilde{x}(t)\cdot R(t,q) \\
	& \le \delta \cdot c(q) \le c(q)
\end{aligned}
\end{equation} 

We also have the following inequalities:
\begin{equation}
\begin{cases}
\frac{\theta^{''}_{f,q}}{\delta \cdot c(q)} \in [0,1]\\
\mathbb{E}[\sum_{f \in F}\frac{\theta^{''}_{f,q}}{\delta \cdot c(q)}] \le 1 
\end{cases}
\end{equation} 
We would like to expect the total secret key generation rate of each relay $q$ wouldn't exceed the key generation capacity $\delta \cdot c(q)$ by a factor of 1+$\epsilon$, where $\epsilon$ is an adjustable parameter. 
\begin{equation}\label{eq:F}
\textbf{Pr}[\bigvee_{q \in Q}\sum_{f \in F} \frac{\theta^{''}_{f,t}}{\delta \cdot c(q)} \ge (1+\epsilon)] \le \Phi
\end{equation}\label{eq:rF}
where $\Phi$ is the network-related function(such as the number of quantum relays N, number of links, etc) and $\Phi \rightarrow 0$ when the QKD network size grows. By applying Lemma \ref{unionbound}, we have the relaxation of Eq.(\ref{eq:F}):
\begin{equation}
\begin{aligned}
&\textbf{Pr}[\bigvee_{q \in Q}\sum_{f \in F} \frac{\theta^{''}_{f,t}}{\delta \cdot c(q)} \ge (1+\epsilon)] \\
&\le \sum_{q \in Q} \textbf{Pr}[\sum_{f \in F} \frac{\theta^{''}_{f,t}}{c(q)} \ge \delta \cdot (1+\epsilon)] \le \Phi
\end{aligned}
\end{equation} 
To satisfy the Eq.(\ref{eq:rF}), we could have the following inequality:
\begin{equation}
\textbf{Pr}[\sum_{f \in F} \frac{\theta^{''}_{f,t}}{\delta \cdot c(q)} \ge  (1+\epsilon)] \le e^{-\frac{\epsilon^2}{2+\epsilon}} \le \frac{\Phi}{N}
\end{equation}
where N is the number of quantum relays $|Q|$in deployed QKD network. Then we can give the following lemma and corresponding proof.

\begin{lemma}\label{lemma:kgc}
	The proposed rounding-based algorithm guarantees that the total key generation rate on any relay will not exceed the key generation capacity of the fractional solution by a factor of $\delta(3+2logN)+1$ with high probability.
	\begin{proof}
		Set $\epsilon = 2 + 2logN$. By applying lemma \ref{chernoff}, it follows:
		\begin{equation}
		\begin{aligned}
		\textbf{Pr}&[\sum_{f \in F} \frac{\theta^{''}_{f,t}}{c(q)} \ge  \delta \cdot (3+2logN)] \\
		& \le e^{-\frac{(2logN+2)^2}{4+2logN}} \\
		& \le  e^{-2logN} = \frac{1}{N^2} 
		\end{aligned}
		\end{equation}
		Combining both path set $T^{'}_f$ and $T^{''}_f$, we have 
		\begin{equation}
		\begin{aligned}
		\textbf{Pr}&[\sum_{f \in F} \frac{\theta_{f,t}}{\c(q)} \ge  \delta \cdot (3+logN) + 1] \\
		& \le \textbf{Pr}[\sum_{f \in F} \frac{\theta^{''}_{f,t}}{c(q)} \ge  \delta \cdot (3+2logN)] \\
		& \le \frac{1}{N^2} 
		\end{aligned}
		\end{equation}
		If we set $\Phi = \frac{1}{N}$, Eq.(\ref{eq:F}) is guaranteeed with $\alpha = 2 + 2logN$ and $\Phi = \frac{1}{N}$
	\end{proof}
\end{lemma}

Like the proof of quantum key generation capacity constraints, we can give the approximation factor of $\lambda$. Let $\lambda'$ be the final result of our algorithm. According to lemma \ref{lemma:kgc}, with respect to the key generation capacity constraints, we obtain the following inequality with high probability.
\begin{equation}
	\frac{\lambda'}{\widetilde{\lambda}} \ge \frac{1}{\delta \cdot (3+2logN)+1}
\end{equation}
As a result, we have the theorem:
\begin{theorem}
	The proposed RSRQ(k) algorithm archives approximation factor of $\frac{1}{\delta \cdot (3+2logN)+1}$ for k-TMQN problem, where N is the number of quantum relays of the deployed QKD network and $0 \le \delta \le 1$.
\end{theorem}


\subsection{Discussion}
We discusss the approximation factor of our algorithm in two special cases. If $k = 1$, that's to say, we only perform QKD through one path for each S-D pair. In this case, it follows that $\sum_{t \in T^{''}_f}\widetilde{x}(t) = \widetilde{\lambda}w(f)$ and $\delta = 1$. Thus, our algorithm achives approximate performance of $O(\frac{1}{logN})$. If we can perform QKD through arbitrarily paths, it follows $\sum_{t \in T^{''}_f}\widetilde{x}(t) = 0$ and $\delta = 0$. In this case, our algorithm can get the optimal result for arbitrarily splittable routing.

However, there are chances that the quantum devices in some relays cannot satisfy all the requirements of S-D pairs.That is, the thoughput factor $\lambda' \le 1$. In such situations, we do need to update the quantum devices in those relays. The addition of quantum devices takes a relatively long time, Besides, we can incrementally update the quantum devices in each relay at a regular time to cope with the increased demand for quantum secret communications in the QKD network. For example, we can update the devices in each relay monthly, based on previous data in recent month.


