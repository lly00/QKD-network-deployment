This section first defines the problem of quantum network topology Control(QNTC) and then gives an approximation algorithm to solve this problem.

\subsection{Problem Definition for QNTC}
In this section, we give the problem definition for QKD network topology control (QNTC). We assume that the cost for each quantum relay is denoted as $\alpha$ and the cost for quantum devices for each quantum channel passing through each relay is denoted as $\beta$. Our goal is to minimum the total cost while deploying the whole QKD network, which is consist of the cost for relays and the cost for devices.

 We use $x_i$ to denote that whether to deploy relay at position $p_i \in P$ ($x_i = 1 $) or not ($x_i = 0$). And $y_i$ denotes the number of quantum channel passing through position $p_i \in P$. Since there should be at least one quantum channel between each S-D pair, there must be at least one available path(includes relays and endpoints) where the physical length of each link is less than $L_{max}$. For any S-D pair $f$, we can find all the candidate paths that satisfy the link length constraint, denoted as $T_f$. $z_j^f$ denotes that if we use the candidate path $t_j^f$ chosen from $T_f$ or not. $w_f$ denotes the quantum bandwidth(or the number of qubit) required by S-D pair $f$ and $v_f$ denotes the data bandwidth required by $f$. $c(e)$ presents the traffic capacity of link $e$.

{\small
	\begin{equation*}
	\min \ \  \alpha\sum_{p \in P}{x_p} + \beta\sum_{p \in P}{y_p}
	\end{equation*}
	\begin{equation}\label{eq:int}
	{S.t.}\begin{cases}
     \sum_{t \in T_f}{z_f^t} = 1, & \forall f \in F \\
     z_f^t \le x_p, & \forall p \in t, \forall t \in T_f, \forall f \in F \\
     \sum_{f \in F}\sum_{p \in t:t \in T_f}{z_f^t \cdot w_f} \le  y_p, & \forall p \in P \\
     \sum_{f \in F}\sum_{e \in t:t \in T_f}{z_f^t \cdot v_f} \le c(e), & \forall e \in E \\
	 x_i \in \{0, 1\}\\
     y_i \in \{1,2,...\}\\
     z_f^t \in \{0,1\}

	\end{cases}
	\end{equation}
}

The first set of equalities denotes each S-D pair $f$ should select one path for quantum key distribution. The second set of inequalities means if a S-D pair $f$ chooses a path $t_j^f$, then all the positions on this path should deploy a relay($x_i$ = 1). The third set of inequalities indicates at most $y_i$ quantum channels can pass through $p_i$. The fourth set of inequalities presents the traffic load on each link e should not exceed c(e).

To find candidate path set $T_f$ for each $f$, firstly we construct a graph with given node set consists of endpoint set $N$ and candidate location set $P$. Then we add a link between any two nodes if the physical length of link between them is less than $L_{max}$. After construction, we can run an algorithm(such as Yen's\cite{Multipaths}) on this graph to get multiple paths $T_f$ for each S-D pair $f$.

\begin{theorem}
  The quantum key distribution network control problem is NP-hard.
\end{theorem}

We can prove that the typical Steiner Minimum Tree with Minimum Steriner Point(SMT-STP) problem is a special case of our problem. Considering each pair of endpoints as a S-D pair and ignoring the quantum device cost, our problem becomes a steiner minimun tree with minimum steiner point problem. Since such a problem is NP-hard, our problem is NP-hard.

% In the scenario of using quantum repeaters, we need contention-free path selection when all nodes communicate concurrently. Quantum resources(such as quantum memories) will be placed in each relay to satisfy this requirement.
%{\small
%	\begin{equation*}
%	\min \ \  \sum_{q_i \in Q}{y_i}
%	\end{equation*}
%	\begin{equation}\label{eq:profairness:int_formulation}
%	{S.t.}\begin{cases}
%	 \sum_{l_j^f \in L_f}{z_j^f} = 1, & \forall f \in F \\
%     \sum_{f \in F,l_j^f \in L_f:q_i \in l_j^f}{z_j^f \cdot w_f} <= \lambda \cdot y_i, & \forall q_i \in Q \\
%	 y_i \in \{1,2,...\}\\
%     z_j^f \in \{0,1\}
%
%	\end{cases}
%	\end{equation}
%}
%$w_f$ denotes the bandwidth needed for each S-D pair $f \in F$ and $y_i$ denotes the number of quantum device(quantum generator, quantum memory) while each device has the ability to handle $\lambda$-bits bandwidth requirements.

\subsection{Problem Complexity Analysis}
The QNTC problem remains challenging even when some constraints are relaxed as mentioned before. We will introduce some special cases in literature to show the universality of our problem.


Steiner Minimum Tree with Minimum Steriner Point with length constraints problem: Given a network topology G=\{D,E\} and a set of point $S$, we use as few points as possible in point set D so that every point in $S$ is connected.

Shortest Path Selection problem: Given a network topology G=\{D,E\}, and a set of S-D pair $F$, we explore the shortest path for each $f \in F$. In this problem, we use hops to present the length of each path.

The main difference from the above two problem to our problem is that our problem is a joint problem of point selection and path selection. However, in steiner minimum tree problem, we only require one available path each S-D pair and in shortest path selection problem, we only search for the shortest path to use as less devices as possible. Typical algorithms for SMT never consider the path lengths for S-D pairs. But in our problem, the path length impacts the number of devices used, so we need to consider both relay selection and path selection at the same time.

This suggests that the QNTC problem is substantially harder than the SMT-MSP problem described above, and thus a different method is needed to find a tight approximation ratio. In the next section, we present a method with approximation performance guarantees.


\subsection{Rounding-Based Deployment Algorithm for QKD Network}
In this section, we propose a rounding-based deployment algorithm for the QKD network topology control problem, described in Alg.\ref{RNTC}. Due to the difficulty of the QNTC problem, the first step obtains the fractional solution for the relaxed QNTC problem. In the second step, we choose one feasible path for each S-D pair using randomized rounding method. Finally. we deploy both relays and devices of all candidate positions based on the paths we choose.

\begin{algorithm}[h]\label{RQTC}
\caption{RQTC:Rounding-based QKD Network Topology Control}
\begin{algorithmic}[1]
\STATE  {\bfseries Step 1: Solving the Relaxed QNTC Problem}
\STATE  Construct a linear optimization program in Eq.\ref{eq:relaxed}
\STATE Obtain the optimal solutions $\widetilde{z}_f^t$
\STATE  {\bfseries Step 2: Selecting Paths for S-D pairs}
\STATE Derive an integer solution $\hat{z_f^t}$ by randomized rounding
\FOR{each S-D pair $f \in F$}
\FOR{each feasible path $t \in T_f$}
\IF{$\hat{z_f^t} = 1 $}
\STATE Appoint a feasible path t for S-D pair f
\ENDIF
\ENDFOR
\ENDFOR
\STATE Select quantum relays and deploy quantum devices according to path selection
\end{algorithmic}
\end{algorithm}

In the first step, the algorithm constructs the relaxed linear problem of Eq.\ref{eq:int}, and solves the relaxed program Eq.\ref{eq:relaxed}. More specifically, the QNTC problem assumes that each S-D pair perform QKD through one feasible path. By relaxing this assumption, the path of each $f$ is permitted to be splittable. We formulate the following linear Program as $LP_1$.

{\small
	\begin{equation*}
	\min \ \  \alpha\sum_{p \in P}{x_p} + \beta\sum_{p \in P}{y_p}
	\end{equation*}
	\begin{equation}\label{eq:relaxed}
	{S.t.}\begin{cases}
     \sum_{t \in T_f}{z_f^t} = 1, & \forall f \in F \\
     z_f^t \le x_p, & \forall p \in t, \forall t \in T_f, \forall f \in F \\
     \sum_{f \in F}\sum_{p \in t:t \in T_f}{z_f^t \cdot w_f} \le  y_p, & \forall p \in P \\
     \sum_{f \in F}\sum_{e \in t:t \in T_f}{z_f^t \cdot v_f} \le c(e), & \forall e \in E \\
     z_f^t \ge 0
	\end{cases}
	\end{equation}
}

Note that the variables $x_p, y_p, z_f^t$ are fractional in Eq.\ref{eq:relaxed} and we can solve it with a linear program solver in linear time. The optimal solution for this linear problem $LP_1$ is denoted by $\widetilde{x}_p,\widetilde{y}_p$ and $\widetilde{z}_f^t$. We use $\lambda$ to present the result $\alpha\sum_{p \in P}{x_p} + \beta\sum_{p \in P}{y_p}$ , so the optimal result is denoted by $\widetilde{\lambda}$. Since $LP_1$ is a relaxation of the QNTC problem, $\widetilde{\lambda}$ is a lower bound result of QNTC. Next, using the randomized rounding method, variable $\hat{z_f^t}$ is set to 1 with the probability of $\widetilde{z}_f^t$ while satisfying $\sum_{t \in T_f}{z_f^t} = 1, \forall f \in F$. It means that the S-D pair f selects path t if $\hat{z_f^t} = 1$.

\subsection{Approximation Performance Analysis}
We analyze the approximate performance of the proposed RQTC algorithm. Assume that the minimum capacity of all the data links is denoted by $c_{min}$. We define variable $\alpha$ as follows:
\begin{equation}\label{alpha}
  \alpha = min\{min\{\frac{c_{min}}{v_f}\}\},min\{\frac{\widetilde{y}_p}{w_f},p \in t,t \in T_f\}\}
\end{equation}
Since RQTC is a randomized algorithm, we compute the expected relay numbers along with the traffic load on data links and quantum relays.


\textbf{Quantum Relay Number.}Considering that if any path $t$ through $p$ is selected, $\hat{x}_p$ is set to 1.We give the expected relay numbers as follows:
\begin{equation}
\mathbb{E}[\hat{x}_p] = 1 - \prod_{p \in t: t \in T_f}{(1-\widetilde{z}_f^t)}
\end{equation}
\begin{theorem}
  The proposed RQTC algorithm achieves the approximation factor of $L$ for quantum relay numbers, where $L$ is the maximum number of feasible paths pass through any position $p \in P$.
  \begin{proof}
  \begin{equation}
  \begin{aligned}
    \mathbb{E}(\hat{x}_p) &= 1 - \prod_{p \in t: t \in T_f,f \in F}{(1-\widetilde{z}_f^t)} \\
    &\le \sum_{p \in t: t \in T_f,f \in F}\widetilde{z}_f^t \\
    &\le L \cdot max_{p \in t: t \in T_f,f \in F}(\widetilde{z}_f^t)
    \end{aligned}
  \end{equation}
  However, $\widetilde{x}_p = max_{p \in t: t \in T_f,f \in F}(\widetilde{z}_f^t)$, thus the approximation factor is $L$.
  \end{proof}
\end{theorem}

We first give two famous theorems for the following probability analysis.
\begin{lemma}\label{chernoff}
(Chernoff Bound):Given n independent variables:$x_1,x_2,...x_n$, where $\forall x_i \in [0,1]$.Let $\mu = \mathbb{E}[\sum_{i=1}^{n}x_i]$. Then $\textbf{Pr}[\sum_{i=1}^{n}x_i \ge (1+\epsilon)\mu] \le e^{\frac{-\epsilon^2\mu}{2+\epsilon}}$, where $\epsilon$ is an arbitrarily positive value.
\end{lemma}

\begin{lemma}
(Union Bound):Given a countable set of n events:$A_1,A_2,...,A_n$, each event $A_i$ happebs with possibility $\textbf{Pr}(A_i)$.Then $\textbf{Pr}(A_i \cup A_2 \cup... \cup A_n) \le \sum_{i=1}^{n}\textbf{Pr}(A_i)$
\end{lemma}



\textbf{Quantum Device Number.} We first give the approximate factor of the quantum channel number passing through of each relay. The first step of the algorithm will derive a fractional solution for $z_f^t$ ,$x_p$ and $y_p$ for the QNTC problem by linear program. Using th  randomized rounding method, for each S-D pair $f \in F$, only one path in $T_f$ will be chosen as its default route. Thus, the traffic load of relay $p$ from S-D pair $f$ is defined as a random variable $y_{p,f}$ as follows:
\begin{definition}\label{def1}
For each $p \in P$ and each $f \in F$, a random varialbe $y_{p,f}$ is defined as:
\begin{equation}\label{ypf}
y_{p,f}=
\begin{cases}
   w_f, with \ probability \ of \ \sum_{p \in t:t \in T_f}\widetilde{z}_f^t \\
   0, otherwise
\end{cases}
\end{equation}

\end{definition}

According to the definition, the expected quantum traffic load in location $p$ is:
\begin{equation}\label{expectedypf}
  \mathbb{E}[\sum_{f \in F}y_{p,f}]= \mathbb{E}[y_{p,f}] = \sum_{f \in F}\sum_{p \in t:t \in T_f}\widetilde{z}_f^t \cdot w_f \le \widetilde{y}_p
\end{equation}

Combining Eq.\ref{expectedypf} and definition of $\alpha$, we have:
\begin{equation}\label{eq:relaxed}
	\begin{cases}
     \frac{y_{p,f} \cdot \alpha}{\widetilde{y}_p} \in [0,1] \\
     \mathbb{E}[\sum_{f \in F} \frac{y_{p,f} \cdot \alpha}{\widetilde{y}_p} \le \alpha]
	\end{cases}
\end{equation}

Then, by applying Lemma \ref{chernoff}, assume $\epsilon$ is an arbitrary positive value. It follows :
\begin{equation}\label{pr1}
  \textbf{Pr}[\sum_{f \in F} \frac{y_{p,f} \cdot \alpha}{\widetilde{y}_p} \ge (1+\epsilon)\alpha] \le e^{\frac{-\epsilon^2\alpha}{2+\epsilon}}
\end{equation}
Now we assume that:
\begin{equation}\label{pr2}
  \textbf{Pr}[\sum_{f \in F} \frac{y_{p,f}}{\widetilde{y}_p} \ge (1+\epsilon)] \le e^{\frac{-\epsilon^2\alpha}{2+\epsilon}} \le \frac{\Phi}{n}
\end{equation}
where $\Phi$ is the function of network-related variables and $\Phi \rightarrow 0$ when the network size $n = |P|$ grows. The solution for Eq.\ref{pr2} is:
\begin{equation}\label{solution}
  \epsilon \ge \frac{log\frac{n}{\Phi}+\sqrt{log^2\frac{n}{\Phi}+8\alpha log\frac{n}{\Phi}}}{2\alpha},n \ge 2
\end{equation}
We give the approximation performance as follows.
\begin{theorem}
The proposed RQTC algorithm achieves the approximation factor of $\frac{3logn}{\alpha}+3$ for quantum device numbers.
\end{theorem}
\begin{proof}
Set $\Phi = \frac{1}{n^2}$,Eq.\ref{pr2} is transformed into:
\begin{equation}
    \textbf{Pr}[\sum_{f \in F} \frac{y_{p,f}}{\widetilde{y}_p} \ge (1+\epsilon)] \le \frac{1}{n^3},where \ \epsilon \ge \frac{3logn}{\alpha}+2
\end{equation}
By applying Lemma \ref{chernoff}, we have
\begin{equation}
\begin{aligned}
    &\textbf{Pr}[\bigvee_{p \in P}\sum_{f \in F} \frac{y_{p,f}}{\widetilde{y}_p} \ge (1+\epsilon)]\\ &\le \sum_{p \in P}\textbf{Pr}[\sum_{f \in F} \frac{y_{p,f}}{\widetilde{y}_p} \ge (1+\epsilon)]\\
    &\le n\cdot \frac{1}{n^3} = \frac{1}{n^2}, \epsilon \ge \frac{3logn}{\alpha}+2
\end{aligned}
\end{equation}
\end{proof}

Then the approximation factor of the algorithm is $\epsilon + 1 = \frac{3logn}{\alpha}+2$

\textbf{Data Link Capacity Constraints.} Next, we analyze the performance of traffic load on data links. Similar to definition \ref{def1}, we give another definition of random variable $\zeta_{e,f}$ as follows:
\begin{definition}
For each link $e \in E$ and each S-D pair $f \in F$, $\zeta_{e,f}$  is defined as:
\begin{equation}
\zeta_{e,f}=
\begin{cases}
     v_f, with \ probability \sum_{e \in t:t \in T_f}\widetilde{z}_f^t \\
     0, otherwise
\end{cases}
\end{equation}
\end{definition}
Similar to the proof before, we can proof that
\begin{equation}\label{pr3}
  \textbf{Pr}[\sum_{f \in F} \frac{\zeta_{e,f} \cdot \alpha}{c(e)} \ge (1+\epsilon)\alpha] \le e^{\frac{-\epsilon^2\alpha}{2+\epsilon}}
\end{equation}
Since there are at most $n^2$ links between $p \in P$, we can assume that:
\begin{equation}\label{pr3}
  \textbf{Pr}[\sum_{f \in F} \frac{\zeta_{e,f}}{c(e)} \ge (1+\epsilon)] \le e^{\frac{-\epsilon^2\alpha}{2+\epsilon}} \le \frac{\Phi}{n^2},n \ge 2
\end{equation}
The solution is
\begin{equation}\label{solution2}
  \epsilon \ge \frac{log\frac{n^2}{\Phi}+\sqrt{log^2\frac{n^2}{\Phi}+8\alpha log\frac{n^2}{\Phi}}}{2\alpha},n \ge 2
\end{equation}
Similarly, by setting suitable values of parameters $\epsilon$ and $\Phi$, we can derive the approximation performance on link capacity constraints.
\begin{theorem}
  The proposed RQTC algorithm achieves the approximation factor of $\frac{4logn}{\alpha}+3$ for data link capacity constraint.
\end{theorem}
\begin{proof}
Set $\Phi = \frac{1}{n^2}$,Eq.\ref{pr3} is transformed into:
\begin{equation}
    \textbf{Pr}[\sum_{f \in F} \frac{\zeta_{e,f}}{c(e)} \ge (1+\epsilon)] \le \frac{1}{n^4},where \ \epsilon \ge \frac{3logn}{\alpha}+2
\end{equation}
By applying Lemma \ref{chernoff}, we have
\begin{equation}
\begin{aligned}
    &\textbf{Pr}[\bigvee_{p \in P}\sum_{f \in F} \frac{\zeta_{e,f}}{c(e)} \ge (1+\epsilon)]\\ 
    &\le \sum_{p \in P}\textbf{Pr}[\sum_{f \in F} \frac{\zeta_{e,f}}{c(e)} \ge (1+\epsilon)]\\
    &\le n^2 \cdot \frac{1}{n^4} = \frac{1}{n^2}, \epsilon \ge \frac{4logn}{\alpha}+2
\end{aligned}
\end{equation}
\end{proof}
\textbf{Approximation Factor.}



